\chapter{前言}
``实分析''一词首先是指实单变量或多变量函数的经典理论: 极限, 连续性, 微分, Riemann 积分, 无穷级数等相关主题. 然而, 时至今日, 其包含了一些更抽象的理论, 这些理论将实变函数论的思想扩展到更普遍的情形, 又为某些具体的``经典''问题带来了新的启示. 此更高等者乃本书之主题.

故本书的受众为已通晓经典实变函数论者. (关于这些主题有很多书, 古老的经典是\cite{rudin1976principles}, 近来最引人入胜者为\cite{korner2004companion}. 此外, 亦有 Steven Krantz 所著之\,《\,MAA\,导论\,》\,\cite{krantz2014guide}与本书同时问世). 据 MAA 导论之理念, 我将以简明之文对该主题加以阐释, 使入门者览其概述, 研习者深其洞见. 基本定义, 主要定理和证明的关键思想都含于其中, 而不含技术细节. 因此, 书中多数正式陈述结果皆有证明思路随后, 其完整程度差异甚大. 很少或没有提供证明的结果分为两类, 命其``命题''或``定理''以区分之. ``命题''之证明较为简易, 宜供读者以为练习. 若称其为定理, 则意味着其证明较为繁冗且不易删减.

自然, 只有在读者有资源来填补空缺时, 这种介绍方能具有效果. 我把我自己的书\cite{folland1999real}作为内容远比本书丰富的标准参考, 只因为我对它最熟悉. 本书所示均已在\cite{folland1999real}证明, 除了某些明确指明其他来源的结果. 再者, \cite{lang2012real}, \cite{royden1988real}和\cite{rudin1987realcomplex}亦涵盖大部分相同材料.

然而, 本书并非\cite{folland1999real}的简写. 教科书作者的一个主要问题是办法若干相互联系的材料线性叙出, 如同小说家或历史学家一般, 且解决方案并非唯一. 我利用MAA导论所提供的机会, 以一种与\cite{folland1999real}中完全不同的方式来安排这些主题. 读者或可比较二者得见一二.

\begin{flushright}
    {\fontspec{lmromancaps10-regular.otf}Gerald B. Folland}\par\kaishu 西雅图, 二〇〇九年四月
\end{flushright}